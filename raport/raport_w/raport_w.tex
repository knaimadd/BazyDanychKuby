\documentclass{article}\usepackage[]{graphicx}\usepackage[]{xcolor}
% maxwidth is the original width if it is less than linewidth
% otherwise use linewidth (to make sure the graphics do not exceed the margin)
\makeatletter
\def\maxwidth{ %
  \ifdim\Gin@nat@width>\linewidth
    \linewidth
  \else
    \Gin@nat@width
  \fi
}
\makeatother

\definecolor{fgcolor}{rgb}{0.345, 0.345, 0.345}
\newcommand{\hlnum}[1]{\textcolor[rgb]{0.686,0.059,0.569}{#1}}%
\newcommand{\hlstr}[1]{\textcolor[rgb]{0.192,0.494,0.8}{#1}}%
\newcommand{\hlcom}[1]{\textcolor[rgb]{0.678,0.584,0.686}{\textit{#1}}}%
\newcommand{\hlopt}[1]{\textcolor[rgb]{0,0,0}{#1}}%
\newcommand{\hlstd}[1]{\textcolor[rgb]{0.345,0.345,0.345}{#1}}%
\newcommand{\hlkwa}[1]{\textcolor[rgb]{0.161,0.373,0.58}{\textbf{#1}}}%
\newcommand{\hlkwb}[1]{\textcolor[rgb]{0.69,0.353,0.396}{#1}}%
\newcommand{\hlkwc}[1]{\textcolor[rgb]{0.333,0.667,0.333}{#1}}%
\newcommand{\hlkwd}[1]{\textcolor[rgb]{0.737,0.353,0.396}{\textbf{#1}}}%
\let\hlipl\hlkwb

\usepackage{framed}
\makeatletter
\newenvironment{kframe}{%
 \def\at@end@of@kframe{}%
 \ifinner\ifhmode%
  \def\at@end@of@kframe{\end{minipage}}%
  \begin{minipage}{\columnwidth}%
 \fi\fi%
 \def\FrameCommand##1{\hskip\@totalleftmargin \hskip-\fboxsep
 \colorbox{shadecolor}{##1}\hskip-\fboxsep
     % There is no \\@totalrightmargin, so:
     \hskip-\linewidth \hskip-\@totalleftmargin \hskip\columnwidth}%
 \MakeFramed {\advance\hsize-\width
   \@totalleftmargin\z@ \linewidth\hsize
   \@setminipage}}%
 {\par\unskip\endMakeFramed%
 \at@end@of@kframe}
\makeatother

\definecolor{shadecolor}{rgb}{.97, .97, .97}
\definecolor{messagecolor}{rgb}{0, 0, 0}
\definecolor{warningcolor}{rgb}{1, 0, 1}
\definecolor{errorcolor}{rgb}{1, 0, 0}
\newenvironment{knitrout}{}{} % an empty environment to be redefined in TeX

\usepackage{alltt}
\usepackage[utf8]{inputenc}
\usepackage{amsfonts}
\usepackage{tgpagella}
\usepackage{graphicx} % Required for inserting images
\usepackage{polski}
\renewcommand*{\figurename}{Rysunek}
\usepackage{nicefrac, xfrac}
\usepackage[margin=1in]{geometry}
\usepackage{hyperref}
\usepackage{xcolor}
\usepackage{amssymb}
\usepackage[bottom]{footmisc}
\usepackage{float}
\IfFileExists{upquote.sty}{\usepackage{upquote}}{}
\begin{document}



Można powiedzieć, że to fajny raport :)).
Cyfry: 1, 2, 3.

\section{Liczba naprawianych pojazdów w każdym miesiącu pracy warsztatu}

{\color{red}[jakiś wstęp do tego]}

\begin{knitrout}
\definecolor{shadecolor}{rgb}{0.969, 0.969, 0.969}\color{fgcolor}\begin{figure}[h]
\includegraphics[width=\maxwidth]{figure/fig_naprawy_miesiecznie-1} \caption[Wykres liczba naprawianych pojazdów w każdym miesiącu pracy warsztatu]{Wykres liczba naprawianych pojazdów w każdym miesiącu pracy warsztatu}\label{fig:fig_naprawy_miesiecznie}
\end{figure}

\end{knitrout}

Wykres \ref{fig:fig_naprawy_miesiecznie} przedstawia liczbę naprawionych pojazdów w każdym miesiącu pracy warsztatu. Najwięcej pojazdów zostało naprawionych w miesiącach: 
listopad 2016,
a było ich 18. Natomiast najmniej przeprowadzonych napraw było w miesiącach:
styczeń 2014, lipiec 2017,
było ich 4. Średnia liczba napraw miesięcznie wynosi 
10.125. 

\section{Profil klienta}

W następnej koljeności zostaną przeanalizawani klienci warsztatu. Zostaną sprawdzone liczności klientów ze względu na różne ich cechy. {\color{red}[Czy warto pisać, że może nam to pomóc stwierdzić, do jakich klientów moglibyśmy spróbować jeszcze dotrzeć.]}

\subsection{Płeć}

Pierwszą cechą jaka zostanie wzięta pod uwagę jest płeć klienta.

\begin{knitrout}
\definecolor{shadecolor}{rgb}{0.969, 0.969, 0.969}\color{fgcolor}\begin{figure}
\includegraphics[width=\maxwidth]{figure/fig_plec-1} \caption[Wykres liczby klientów przy podziale ze względu na płeć]{Wykres liczby klientów przy podziale ze względu na płeć}\label{fig:fig_plec}
\end{figure}

\end{knitrout}

Na wykresie słupkowym \ref{fig:fig_plec} są zaprezentowane liczności klientów przy podziale ze względu na płeć. Więcej klientów warsztatu należy do grupy kobiet. Grupa kobiet jest około 1.024
razy większa od grupy mężczyzn, a zatem różnica jest niewielka.

\subsection{Wiek}

Zostanie również przeanalizowany rozkład wieku klientów warsztatu.





\begin{knitrout}
\definecolor{shadecolor}{rgb}{0.969, 0.969, 0.969}\color{fgcolor}\begin{figure}
\includegraphics[width=\maxwidth]{figure/fig_wiek-1} \caption[Wykres pudełkowy wieku klientów]{Wykres pudełkowy wieku klientów}\label{fig:fig_wiek}
\end{figure}

\end{knitrout}

Rysunek \ref{fig:fig_wiek} przedstawia wykres pudełkowy wieku klientów warsztatu. Widać, że mediana wynosi [mediana], natomiast pierwszy kwartyl wynosi {\color{red}[Q1]}, trzeci kwartyl natomiast {\color{red}[Q3]}. {\color{red}[Mogłabym coś napisać, który jest bardziej oddalony od mediany]}. Najmłodszy klient warsztatu ma {\color{red}[najmniejszy wiek]}, natomiast najstarszy klient ma {\color{red}[najwyższy wiek]}.

\begin{table}[H]
\centering
\begin{tabular}{|c|c|} \hline
średnia & odchylenie standardowe \\ \hline
45.15 & 10.9 \\ \hline
\end{tabular}
\caption{Tabela z ...}
\label{tab_wiek}
\end{table}

Kilka miar, których nie da się odczytać z wykresu, zostało przedstawionych w tabli \ref{tab_wiek}. Można zatem odczytać, że średnio klieci mają 45.15, a odchylenie standardowe wieku wynoski 10.9. {\color{red}[Zdanie że średnia jest mniejsza/podobna/większa a zatem jakaś skośność, co potwierdza współczynnik skośności]}

\subsection{Miasto}
\begin{knitrout}
\definecolor{shadecolor}{rgb}{0.969, 0.969, 0.969}\color{fgcolor}\begin{kframe}
\begin{verbatim}
##      miasto liczność
## 1   Wrocław      719
## 2      Łódź       27
## 3    Poznań       23
## 4    Kraków       20
## 5     Opole       19
## 6 Bydgoszcz       17
\end{verbatim}
\end{kframe}
\end{knitrout}

[opis tabelki]

\subsection{Karta lojalnościowa}
\begin{knitrout}
\definecolor{shadecolor}{rgb}{0.969, 0.969, 0.969}\color{fgcolor}
\includegraphics[width=\maxwidth]{figure/unnamed-chunk-5-1} 
\end{knitrout}

\section{Jak różne cechy pojazdów wpływają na ich sprzedaż / nasz zarobek?}

[tabelka z miarami ???]





\end{document}
